\section{Introduction}

One major assumption in the majority of digital VLSI circuits produced
today, is the notion of a discrete and common time. This is provided
by one or more clocks. This is an important simplification that makes
reasoning about digital circuits easyer. However, as circuits grow in
size and complexity, the task of providing a clock-signal
simultaneously to all flip-flops in the circuit becomes an
increasingly difficult task.

When designing a digital circuit for solving a computable problem, it
is usual to split the problem into multiple combinatorical circuits
with memory elements between them. The value of a memory element is
then determined by the value of the previous memory element modified
by a combinatorical circuit. However, this combinatorical circuit has
a non-zero delay, and there is usually no reliable way to assure that
the output value has been completly calculated. With clocked circuits,
the completeness of a calculation is guaranteed by assuming worst case
conditions and setting the clock-frequency accordingly.

As digital circuits have grown, designing has decreasingly become a
problem of specification, and increasingly become problem of managing
complexity. Hardware description languages, such as VHDL and Verilog,
is maybe the most important tools to manage complexity for digital
designers.

In this project, I will investigate options for synthesizing digital
circuits without clocks. A prerequisite for such solutions is a
high-level language that ultimatly can be implemented in silicon. I
will also attempt to implement an encryption-circuit for the
AES-standard with one of theese technologies.

The motivation for investigating clockless is that the clock, while
conceptually simple, imposes increased complexity when the circuits
grow; the clock can ultimatly end up complicating the design instead
of simplifying it. Modern CAD tools employ complex and proprietary
techniques for distributing the clock througout the circuit.

Clockless circuits also have different and interesting characteristics:
\begin{itemize}
\item The clock in a conventional design is one of the main contributer to
power-consumption. 

\item The simultanity of the clock make the circuits exhibit noticeable
spikes in the electromagnetic spectrum.

\item The whole problem of clock domain crossing is sidestepped, allowing a
higher degree of modularity and reuseability.

\item Clockless designs are more robust to process techonlogies and
operating conditions. Clockless circuits operates at its highest
possible speed, and adapts to doping variations, bending and stretching in
contemporary strechable circuits, temperature and voltage.

\item Clockless circuits does not require massive, ``cheap'', metallic
fan-outs provided by the silicon-techonlogy today. Future technologies
for implementing logic, such as ``rapid single flux quantum'', does not
provide a similar simple fan-out mechanism. 
\end{itemize}

When mapping high-level description hardware-languages to silicon,
there must also be implemented a form of completion-detection for the
combinatorical circuits to drive the memory-elements instead of the
clock. I will outline multiple of these techniques in this project,
but common for them all is that they impose overhead in the form of
increased area, power and complexity.

Another issue with clockless circuits is that it is a relatively young
discipline. There is not an industry-wide adoption, nor experience
with clockless design. Testing is one obstacle that has not been
suffiently researched. 

\subsection{AES}

Reason for choice of AES for this thesis. Well specified.

About the rijandael algorithm.

Clockless cryptography.
