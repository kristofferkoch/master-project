\section{Introduction}

In this project, I will investigate options for synthesizing digital
circuits without clocks. A prerequisite for such a solution is a
high-level language that ultimately can create circuits for
implementation in silicon. I will also implement an encryption circuit
for the AES-standard with one of these technologies, and attempt to
identify challenges and potential with such a solution.

One major assumption in the majority of digital VLSI circuits produced
today, is the notion of a discrete and common time. This is provided
by one or more clocks. This is an important simplification that makes
reasoning about digital circuits easier. However, as circuits grow in
size and complexity, it is ``well known'' \cite[pp. 5]{sparso} that
the task of providing a clock-signal simultaneously to all flip-flops
in the circuit becomes an increasingly difficult task.

When designing a digital circuit for solving a computable problem, it
is usual to split the problem into multiple combinational circuits
with memory elements interspersed (XXX Figure!). The value of a memory
element is then determined by the value of the previous memory element
modified by a combinational circuit. However, this combinational
circuit has a non-zero delay, and there is usually no reliable way to
determine when the output value has been completely calculated. With
clocked circuits, the completeness of a calculation is guaranteed by
assuming worst case conditions and setting the clock-frequency
accordingly.

As digital circuits have grown, designing has increasingly become
problem of managing complexity \cite{flynn2009deep}. To make
specifications with varying levels of abstractions in hardware
description languages, such as VHDL and Verilog, is crucial for
digital designers to manage complexity.

The motivation for investigating clockless is that the clock, while
conceptually simple, imposes increased complexity when the circuits
grow; the clock can ultimately end up complicating the design instead
of simplifying it. Clockless circuits also have different and
interesting characteristics: \begin{itemize}

\item The clock in a conventional design is one of the main
  contributors to power-consumption \cite{tiwari1998reducing}.

\item The simultaneity of the clock make the circuits exhibit
  noticeable spikes in the electromagnetic spectrum.

\item The whole problem of clock domain crossing is sidestepped,
  allowing a higher degree of modularity and reuseability. [XXX Sparsø
    ref: 67 (1950!), 57, 108, 97, 94]

\item Clockless designs are more robust to process technologies and
  operating conditions. The speed of clockless circuits automaticly
  adapts to doping variations, bending and stretching in stretchable
  circuits, temperature and voltage.

\item Clockless circuits does not require massive, ``cheap'', metallic
  fan-outs (the clock) provided by the silicon-technology today. Future
  technologies for implementing logic, such as ``rapid single flux
  quantum'', may not provide a similar simple fan-out mechanism.
\end{itemize}

When mapping high-level description hardware-languages to silicon,
there must also be implemented a form of completion-detection for the
combinational circuits to drive the memory-elements instead of the
clock. I will outline some of these techniques in this project, but
common for them all is that they impose overhead in the form of
reduced speed and increased area, power and complexity.

Another issue with clockless circuits is that it is a relatively young
discipline. There is not an industry-wide adoption, nor experience
with clockless design. Testing is one obstacle that has not been
sufficiently researched. XXX. Write more
