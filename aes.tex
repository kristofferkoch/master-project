\section{The Advanced Encryption Standard (AES)}

On September 12, 1997, the United States' National Institute of
Standards and Technology (NIST) announced an open competition to
develop a new symmetric-key block cipher algorithm to replace the
ageing DES (Data Encryption Standard). The demands was that the AES
should be an unclassified, public, and royalty-free symmetric-key
encryption algorithm supporting block sizes of 128-bits, key sizes of
128-, 192- and 256-bits.

On October 2, 2000, NIST announced that it had selected
Rijandael~\cite{rijandael}, designed by two Belgium cryptographers:
Daemen and Rijmen, as the AES.

\subsection{Overview of the algorithm}

Rijandael is a symmetric-key block-cipher algorithm. This means that
encryption is defined as $c \leftarrow E_k(m)$, and decryption as $m
\leftarrow D_k(c)$, given a key $k$, a message $m$ and a ciphertext
$c$. Rijandael supports variable block and key sizes, but I will
confine the description and implementation to block and key size to
128~bits. This will not cause a big loss of generality to the working
principle. I will also limit the implementation to encryption-only.

The Rijandael cipher segments a 128-bit message into 16 bytes,
represented by a $4 \times 4$ matrix:

\begin{equation}
  m = \begin{pmatrix}
    m_0 & m_4 & m_8 & m_12 \\
    m_1 & m_5 & m_9 & m_13 \\
    m_2 & m_6 & m_10 & m_14 \\
    m_3 & m_7 & m_11 & m_15
    \end{pmatrix}
\end{equation}

This is referred to as the ``State'' througout the algorithm, and it
is permutated 10 rounds in the algorithm. For larger key sizes, the
number of rounds should be increased. A round is composed of four
transformations described below:

\begin{verbatim}
Round(State, RoundKey):
   SubBytes(State)
   ShiftRows(State)
   MixColumns(State)
   AddRoundKey(State, RoundKey)
\end{verbatim}

The final round is slightly different: it excludes the MixColumns
stage. ``RoundKey'' is derived from the key by the key expansion
scheme also described below. For decrypting, given the correct key,
there exists inverse functions for each round.

The Rijandael cipher work in a finite field. The field is realized as
all polynomials modulo the irreducivle polynomial $f(x) = x^8 + x^4 +
x^3 + x + 1$ over $\mathbb{F}_2$. This is called the ``Rijandael
field'', and $\mathbb{F}_{w^8}$ is often used to denote the field with
256 element. Each element can also be represented as a 8-bit byte. All
operations on elements in this field results in an element within the
field. The field supports addition and multiplication between two
field elements.

\subsubsection{SubBytes}

$y = A x^-1 + b$

\subsubsection{ShiftRows}



\subsubsection{MixColumns}

\subsubsection{AddRoundKey}

\subsubsection{Key expansion}

\subsection{Modes of operation}

\subsection{Implementation details for VLSI implementation}

\subsubsection{S-box implementations}

\subsubsection{Word width}


\subsubsection{Pipelining}


