\section{The Advanced Encryption Standard (AES)}

On September 12, 1997, the United States' National Institute of
Standards and Technology (NIST) announced an open competition to
develop a new symmetric-key block cipher algorithm to replace the
ageing DES (Data Encryption Standard). The demands was that the AES
should be an unclassified, public, and royalty-free symmetric-key
encryption algorithm supporting block sizes of 128-bits, key sizes of
128-, 192- and 256-bits.

On October 2, 2000, NIST announced that it had selected
Rijandael~\cite{rijandael}, designed by two Belgium cryptographers:
Daemen and Rijmen, as the AES.

%As of today, multiple processors include native support for
%acceleration of AES-specific operations. NIST has also called for the
%development of a new hash-function, and 4 of 14 of the remaining
%candidates rely on native AES-acceleration to acheive performance.

\subsection{Overview of the algorithm}

Rijandael is a symmetric-key block-cipher algorithm. This means that
encryption is defined as $c \leftarrow E_k(m)$, and decryption as $m
\leftarrow D_k(c)$, given a key $k$, a message $m$ and a ciphertext
$c$. Rijandael supports variable block and key sizes, but I will
confine the description and implementation to block and key size to
128~bits. This will not cause a big loss of generality to the working
principle. I will also limit the implementation to encryption-only.

The Rijandael cipher segments a 128-bit message into 16 bytes,
represented by a $4 \times 4$ matrix:

\begin{equation}
  m = \begin{pmatrix}
    m_0 & m_4 & m_8 & m_{12} \\
    m_1 & m_5 & m_9 & m_{13} \\
    m_2 & m_6 & m_{10} & m_{14} \\
    m_3 & m_7 & m_{11} & m_{15}
    \end{pmatrix}
\end{equation}

This is referred to as the $State$ througout the algorithm, and it
is permutated 10 rounds in the algorithm. For larger key sizes, the
number of rounds should be increased. A round is composed of four
transformations described below:

\begin{eqnarray*}
&Round&(State, RoundKey) \{\\
  & &SubBytes (State)\\
  & &ShiftRows (State)\\
  & &MixColumns (State)\\
  & &AddRoundKey (State, RoundKey)\\
&\}&
\end{eqnarray*}

The final round, $FinalRound$, is slightly different: it excludes the
MixColumns stage. $RoundKey$ is derived from the key by the key
expansion scheme also described below. For decrypting, given the
correct key, there exists inverse functions for each round.

The Rijandael cipher work in a finite field. The field is realized as
all polynomials modulo the irreducivle polynomial $f(x) = x^8 + x^4 +
x^3 + x + 1$ over $\mathbb{F}_2$. This is called the ``Rijandael
field'', and $\mathbb{F}_{w^8}$ is often used to denote the field with
256 element. Each element can also be represented as a 8-bit byte. All
operations on elements in this field results in an element within the
field. The field supports addition and multiplication between two
field elements.

\subsubsection{$SubBytes$}

$y = A x^{-1} + b$, where $
  A =
  \begin{pmatrix}
    1 & 0 & 0 & 0 & 1 & 1 & 1 & 1 \\
    1 & 1 & 0 & 0 & 0 & 1 & 1 & 1 \\
    1 & 1 & 1 & 0 & 0 & 0 & 1 & 1 \\
    1 & 1 & 1 & 1 & 0 & 0 & 0 & 1 \\
    1 & 1 & 1 & 1 & 1 & 0 & 0 & 0 \\
    0 & 1 & 1 & 1 & 1 & 1 & 0 & 0 \\
    0 & 0 & 1 & 1 & 1 & 1 & 1 & 0 \\
    0 & 0 & 0 & 1 & 1 & 1 & 1 & 1
  \end{pmatrix}$ and $
  b = 
  \begin{pmatrix}
    1\\1\\0\\0\\0\\1\\1\\0
  \end{pmatrix}$


\subsubsection{$ShiftRows$}

\begin{equation}
  \begin{pmatrix}
    s_0 & s_4 & s_8    & s_{12} \\
    s_1 & s_5 & s_9    & s_{13} \\
    s_2 & s_6 & s_{10} & s_{14} \\
    s_3 & s_7 & s_{11} & s_{15}
  \end{pmatrix} 
  \rightarrow
  \begin{pmatrix}
    s_0    & s_4    & s_8    & s_{12} \\
    s_{13} & s_1    & s_5    & s_9 \\
    s_{10} & s_{14} & s_2    & s_6 \\
    s_7    & s_{11} & s_{15} & s_3    
    \end{pmatrix}
\end{equation}

\subsubsection{$MixColumns$}

For each of the columns in the state as $
\begin{pmatrix}
  s_0\\s_1\\s_2\\s_3
\end{pmatrix}$

\subsubsection{$AddRoundKey$}

\subsubsection{$Key expansion$}


\subsection{Implementation details for VLSI implementation}

\subsubsection{$SubBytes$ implementations}

Usually, the $SubBytes$ routine and its inverse is implemented with a
table lookup with 256 entries. In hardware, this means some kind of a
ROM, with an unbreakable latency and the inherent impossibility for
pipelining. In \cite{csbox} a method is described for decomposing the
$2^8$ fields into smaller $2^4$ fields for calculating the inverse
combinationally.

\subsubsection{Word width}

\subsubsection{Resource sharing}


\subsubsection{Pipelining and mode of operation}

While it is possible to pipeline the $Round$ procedure by itself, and
even $SubBytes$ itself when implemented combinationally as in
\cite{csbox}, it can also be beneficial to unroll multiple $Round$s in
series as shown in figure~\ref{fig:unroll}. I will here describe 3
modes of operation to illustrate when unrolling can be beneficial.

When encrypting a plaintext with a length over 128 bits, the most
straightforward way is to split the plaintext into 128-bit blocks and
encrypt them individually. This is called the electronic codebook mode
of operation, or ECB. As the blocks can be encrypted and decrypted
independently, this method allows pipelining with unrolled
$Round$s. However, as the AES-algorithm deterministicly yields the
same output given the same input (key and data), it allows an attacker
to guess the ciphertext by trial-and-error. Figure~\ref{fig:ecbpict}
shows how ECB reveals data-patterns from the plaintext.

The output feedback (OFB) mode of operation, described in
figure~\ref{fig:ofb}, while not suffering from the same weakness as
the ECB mode, does not benefit from unrolling, as the preceeding
ciphertext is needed to encrypt the current plaintext. The OFB mode
generates a pseudo random (but unguessable, not given the key) string
that is XORed with the plaintext to encrypt, similar to the Vernam
one-time-pad cipher \cite{vernam}.

The only mode of operation considered secure, to my knowledge, that
benefit from loop unrolling, is the counter (CTR) mode. In this mode,
similar to the OFB mode, a pseudorandom string is generated simpy by
encrypting an integer corresponding to the current block number, that
is $c_i \leftarrow m_i \oplus E_k(i)$, and similarily $m_i \leftarrow
c_i \oplus E_k(i)$. Whithout feedback, the CTR mode allows random
access during encryption and decryption (for e.g. full disk
encryption), and blocks can be processed in paralell.

