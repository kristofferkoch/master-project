\section*{Assignment description}

Challenging the frontiers of ultra low-energy design requires
innovative low-energy contributions from all levels and aspects of the
design and toolflow. Software architecture, digital system-level
architecture, low-level digital design choices, backend techniques
such as clock gating strategies, toggle information feedback and
operand isolation, process technology and place-and-route techniques
are all critical factors in the total digital world energy consumption
budget.

Some fundamental paradigms of main-stream digital design flows have
not changed for decades, such as the presence of a clock. Research
into self-timed techniques (we are avoiding the term ‘asynchronous’
here because an asynchronous design often refers to a design with
multiple asynchronous clocks) has progressed steadily over the last
two decades. Self-timed design promises major improvements in energy
consumption and noise emission compared to traditional clocked
synchronous designs.

A major hurdle that had to be solved to enable large scale
commercialized of such circuits was testability. Until recently,
self-timed digital circuits were not scanable, and therefore
traditional production testing and low-cost mass production was not
possible.

A range of different approaches have been developed for the design and
synthesis of self-timed circuits. The Netherlands based company
Handshake Solutions (www.handshakesolutions.com) is one example of a
tool and IP provider for self-timed circuit design.

The focus of this thesis is to conduct a pre-study into the viability
of using self-timed design techniques for low-cost ultra-low energy
designs such as the EFM32 microcontroller family. Such techniques will
not be applicable for the entire digital part of a MCU, but for
certain peripheral and core modules. In this thesis, a study into
converting a synchronous Advanced Encryption Standard (AES) peripheral
module into a self-timed and scanable peripheral module is the primary
objective. Various self-timed techniques should be explored and
compared to a traditional synchronous design.
