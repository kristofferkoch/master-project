\section{Sourcecode}

\lstset{ %
basicstyle=\footnotesize,       % the size of the fonts that are used for the code
numbers=left,                   % where to put the line-numbers
numberstyle=\footnotesize,      % the size of the fonts that are used for the line-numbers
stepnumber=1,                   % the step between two line-numbers. If it's 1 each line will be numbered
numbersep=5pt,                  % how far the line-numbers are from the code
showspaces=false,               % show spaces adding particular underscores
showstringspaces=false,         % underline spaces within strings
showtabs=false,                 % show tabs within strings adding particular underscores
frame=single,                % adds a frame around the code
tabsize=3,                % sets default tabsize to 2 spaces
captionpos=b,                   % sets the caption-position to bottom
breaklines=true,                % sets automatic line breaking
breakatwhitespace=false,        % sets if automatic breaks should only happen at whitespace
escapeinside={\%*}{*)}          % if you want to add a comment within your code
}


\subsection{Test fixture, test-simple.balsa}
\lstinputlisting{naive-AES-balsa/test-simple.balsa}

\subsection{aes.balsa}
\lstinputlisting{naive-AES-balsa/aes.balsa}

\subsection{sbox.balsa}
\lstinputlisting{naive-AES-balsa/sbox.balsa}

\subsubsection{affine.balsa}
\lstinputlisting{naive-AES-balsa/affine.balsa}

\subsubsection{inversion.balsa}
\lstinputlisting{naive-AES-balsa/inversion.balsa}

\subsubsection{gf4.balsa}
\lstinputlisting{naive-AES-balsa/gf4.balsa}


\subsection{mixcolumn.balsa}
\lstinputlisting{naive-AES-balsa/mixcolumn.balsa}

\subsection{gfdouble.balsa}
\lstinputlisting{naive-AES-balsa/gfdouble.balsa}
