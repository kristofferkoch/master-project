\section{Theory}

When designing clockless circuits, we need a substitute for the clock
to assure data validity, and flow for the memory elements. This is
done with handshaking, usually implemented with a request-acknowledge
protocol as shown in figure~\ref{fig:hanshake}. The handshaking
provides a local clock for the storage elements\footnote{Usually
  single latches, as master-slave latches is unneeded.}.

There are multiple ways to implement handshaking. This project will
focus on high-level synthesis that is agnostic on the underlying
protocol (it can even be clocked). 

\subsection{2-phase and 4-phase handshaking}


\subsection{Push and pull channels}

\subsection{Indication and the muller-C element}

\subsection{Bundled-data}

When using bundled-data, values are represented by conventional
boolean levels, and the handshaking is implemented by bundeling the
request and acknowledge signals with the data as shown in
figure~\ref{fig:bundeled}. Bundled data is also refered to as
single-rail.

As binary data usually has no way of indicating completion, matching
delays has to be inserted to maintain correct behaviour as shown in
figure~\ref{fig:bundeled_delay}.

\subsection{Dual rail}

If the signal is encoded into a representation using two wires per
bit, one for each value; logic 1 (true) and logic 0 (false). This
redundant representation provides the possibility of a variable to
have an ``empty'' value. 

By carefully designing the combinational blocks without hazards, it is
possible to detect when values are completely computed. An example of
a circuit for completion detection is shown in figure
\ref{fig:completion}. By using this completion-detection for the
request (or acknowledge) signal, the circuit can be made delay
insensitive.

\subsection{Summary}


%\begin{quote}
%A register may input and store a new data token from its predecessor
%if its successor has input and stored the data token that the register
%was previously holding
%\end{quote}

